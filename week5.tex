\documentclass[PlasmaNotes.tex]{subfiles}
\begin{document}

\section{Week 5. SPAAAAAAAAAAAAAAAAAAAAAAACE}

With this out of the way,

\subsection{The Sun}

\subsubsection{Solar properties}
\begin{itemize}
\item Mean distance $d_0 = 1.5e11 m = 1 AU$
\item Mass $M_0=2e30 kg$, typical for \emph{main sequence stars}
\item Radius $R_0 = 7e8 m$. Sharply defined (photosphere)
\item Luminosity $L_0=3.84e26 W$. Can use $E=mc^2$ to get $\d{m}{t} = 10^9 kg/s$ 
\item About 80\% hydrogen
\end{itemize}

Logarithmic star luminosity grows linearly with logarithmic mass

\[ L/L_0 = (M/M_0)^{\alpha} \]
\[ 3 < \alpha < 4\]

A star $10$ times heavier will burn mass $1000$ times faster. Lower lifetime.

\subsubsection{Hydrostatic equilibrium}

Assume: Sun is perfectly spherical. Chop it up in spherical shells, radial distance $r$, thickness $dr$. Local matter density $\rho$. $M_r$ - total mass inside shell.

Use force balance to get relation between acceleration, gravity and radial change in pressure (pressure forces).

Set acceleration to zero for hydrostatic equilibrium and solar death rays to kill for maximum awesome, 

\[ \d{P}{r} = - G M_r \rho /r^2 \]

Integrate shells ($4 \pi r^2$) from $0$ to $R_0$.

\[3 \avg{P} V = \int_0^{R_0}{\rho G M_0 4 \pi r dr} \]

Right side is gravitational energy. Assume volume is filled with a plasma, N protons, N electrons, temperature T.

\[ \avg{P} = 2 N k_b T/V \]

Integrating, we get an average temperature $2.3e6 K$. Maximum can reach $1.6e7 K$

\subsubsection{Proton proton chain reaction (pp I chain)}

Total energy release - $26.22 MeV$.

\subsubsection{Solar luminosity}

Energy density proportional to $T^4$. Total radiative energy you get by multiplying by volume. Luminosity can be gotten as total energy over average time to reach solar surface.

Simplest assumption for a photon going straight from solar center gets you a photon getting there in $500s$.

The plasma is not transparent! Photons get absorbed, scattered, reemitted...

\begin{itemize}
\item Free free absorption (inverse bremsstrahlung)
\item Bound free absorption (photoionization)
\item bound bound absorption (photoexcitation)
\item electron scattering (Thompson, Compton scattering)
\end{itemize}

Mean free path for Thomson scattering in solar interior is $2 cm$, that's awfully low. This means photons don't go in a straight line but diffuse in a random walk

\[ \avg{R^2} = D \tau_{ph} \]

Under this assumption, a photon takes $10e8 y$ to escape. And this fits the experimental data..... Whoa.

We get the result that luminosity scales as $M^3$. Just what we began with for main sequence stars!


\subsection{The solar cycle. Solar magnetic fields}

Hydrodynamic dynamo process generates the magnetic fields

\subsubsection{Sunspots}

\textbf{Sunspots} - dark patches through which magnetic fields emerge from the interior. The number of sunspots is periodic.

International Sunspot Number, an average value from observatories over the globe.

This is strikingly periodic!

Sunspots larger than the Earth. AB/944

Dark region - umbra, lighter around it - penumbra. Umbrae are characterized by $\%$ of hemisphere surface, and latidude positions.

Solar rotation time: 1 month approximatley

\subsubsection{The butterfly diagram}

Sunspots range from $-30$ to $+30$ degrees in latitude. At the beginning of each cycle, sunspots occur at maximum values and converge towards the equator (which, however, stays free of sunspots)

\subsubsection{Magnetic field measurements}

Started in 1908 by Hale. Zeeman splitting implied magnetic fields about $ 0.2 T$.

Binary groups - fields leaving and entering Sun, closed field loops. Groups on northern hemisphere have symmetric groups on the southern hemisphere, with switched order of entering/leaving sun by mag fields.

\subsubsection{Hale's law}

Leading sunspot (in rotation) tends to be closer to the equator

Between periods, polarities of sunspots reverse!

\subsubsection{Global magnetic field structure}

At solar minimum, polar fields about 10 Gauss - much lower than in sunspots. With no sunspots, we'd get a dipole field. 

At solar maximum, multipolar field, lots of twisted lines leaving and entering solar surface

At next solar minimum, reversed polarity.

\end{document}
