\documentclass[PlasmaNotes.tex]{subfiles}
\begin{document}
\setcounter{section}{4}

\section{Week 5. SPAAAAAAAAAAAAACE, with Ivo Furno}

\subsection{The Sun}

\subsubsection{Solar properties}
\begin{itemize}
\item Mean distance $d_0 = 1.5e11 m = 1 AU$
\item Mass $M_0=2e30 kg$, typical for \emph{main sequence stars}
\item Radius $R_0 = 7e8 m$. Sharply defined (photosphere)
\item Luminosity $L_0=3.84e26 W$. Can use $E=mc^2$ to get $\d{m}{t} = 10^9 kg/s$ 
\item About 80\% hydrogen
\end{itemize}

Logarithmic star luminosity grows linearly with logarithmic mass

\[ L/L_0 = (M/M_0)^{\alpha} \]
\[ 3 < \alpha < 4\]

A star $10$ times heavier will burn mass $1000$ times faster. Lower lifetime.

\subsubsection{Hydrostatic equilibrium}

Assume: Sun is perfectly spherical. Chop it up in spherical shells, radial distance $r$, thickness $dr$. Local matter density $\rho$. $M_r$ - total mass inside shell.

Use force balance to get relation between acceleration, gravity and radial change in pressure (pressure forces).

Set acceleration to zero for hydrostatic equilibrium and solar death rays to kill for maximum awesome, 

\[ \d{P}{r} = - G M_r \rho /r^2 \]

Integrate shells ($4 \pi r^2$) from $0$ to $R_0$.

\[3 \avg{P} V = \int_0^{R_0}{\rho G M_0 4 \pi r dr} \]

Right side is gravitational energy. Assume volume is filled with a plasma, N protons, N electrons, temperature T.

\[ \avg{P} = 2 N k_b T/V \]

Integrating, we get an average temperature $2.3e6 K$. Maximum can reach $1.6e7 K$

\subsubsection{Proton proton chain reaction (pp I chain)}

Total energy release - $26.22 MeV$.

\subsubsection{Solar luminosity}

Energy density proportional to $T^4$. Total radiative energy you get by multiplying by volume. Luminosity can be gotten as total energy over average time to reach solar surface.

Simplest assumption for a photon going straight from solar center gets you a photon getting there in $500s$.

The plasma is not transparent! Photons get absorbed, scattered, reemitted...

\begin{itemize}
\item Free free absorption (inverse bremsstrahlung)
\item Bound free absorption (photoionization)
\item bound bound absorption (photoexcitation)
\item electron scattering (Thompson, Compton scattering)
\end{itemize}

Mean free path for Thomson scattering in solar interior is $2 cm$, that's awfully low. This means photons don't go in a straight line but diffuse in a random walk

\[ \avg{R^2} = D \tau_{ph} \]

Under this assumption, a photon takes $10e8 y$ to escape. And this fits the experimental data..... Whoa.

We get the result that luminosity scales as $M^3$. Just what we began with for main sequence stars!


\subsection{The solar cycle. Solar magnetic fields}

Hydrodynamic dynamo process generates the magnetic fields

\subsubsection{Sunspots}

\textbf{Sunspots} - dark patches through which magnetic fields emerge from the interior. The number of sunspots is periodic.

International Sunspot Number, an average value from observatories over the globe.

This is strikingly periodic!

Sunspots larger than the Earth. AB/944

Dark region - umbra, lighter around it - penumbra. Umbrae are characterized by $\%$ of hemisphere surface, and latidude positions.

Solar rotation time: 1 month approximatley

\subsubsection{The butterfly diagram}

Sunspots range from $-30$ to $+30$ degrees in latitude. At the beginning of each cycle, sunspots occur at maximum values and converge towards the equator (which, however, stays free of sunspots)

\subsubsection{Magnetic field measurements}

Started in 1908 by Hale. Zeeman splitting implied magnetic fields about $ 0.2 T$.

Binary groups - fields leaving and entering Sun, closed field loops. Groups on northern hemisphere have symmetric groups on the southern hemisphere, with switched order of entering/leaving sun by mag fields.

\subsubsection{Hale's law}

Leading sunspot (in rotation) tends to be closer to the equator

Between periods, polarities of sunspots reverse!

\subsubsection{Global magnetic field structure}

At solar minimum, polar fields about 10 Gauss - much lower than in sunspots. With no sunspots, we'd get a dipole field. 

At solar maximum, multipolar field, lots of twisted lines leaving and entering solar surface

At next solar minimum, reversed polarity.

\subsection{Dynamo - from plasma flow to magnetic fields}

Mechanical energy (fluid movement) converted into stretching and twisting B lines

Faraday rotor - homopolar disk dynamo. Metal disk rotating at constant angular velocity. Wire transfers charge to central rod.

Above a critical value of angular velocity, there's flux amplification and dynamo action.

Dynamos result from asymmetries of internal motion - fluid movement inside the Sun

Theorems:
\begin{enumerate}
\item there are no 2D dynamo fields. It always depends on all 3 coordinates.
\item Can't make a dynamo with a toroidal flow.
\item Can't have a field vanishing at infinity (e.g. physical) for axisymmetric
\item High degree of symmetry stops dynamo action: stationary axisymmetric, stationary centrally symmetric, planar velocity field
\end{enumerate}

\subsubsection{Surface rotation}
Non-uniform rotation of solar surface - equator rotates with period of 25 days. Closer to the pole, period increases. The Sun is not a solid body, but a miasma of incandescent plasma, of course.

\subsubsection{Helioseismology}

Acoustic waves in the sun allows to determine rotation rate relation to depth

Up to $0.7 R_0$ the sun rotates as a solid body

There's a $0.04R_0$ area called the \textbf{tachocline} where the sun experiences extreme shear forces

\subsubsection{Meridional circulation}
Flow from equator to poles

\begin{center}
I skipped making notes in this section. If anyone wants to fill this part out, feel free to!
\end{center}

\subsection{Reconnection: from magnetic fields to flows}

\begin{center}
Likewise, I skipped making notes in this chapter.
\end{center}

\subsection{How does the solar wind blow?}

We'll use an MHD description to describe the solar wind.

Cometary tails pointing away from Sun regardless of velocity imply ionized gas pushed away from the comet by the solar wind.

Example - Hale-Bopp comet. White tail - light pressure, radially away from Sun. Blue tail - not radially away from Sun, points in another direction - it's solar wind. Doesn't necessarily have to blow radially, right?

Solar wind speed measured for the first time in 1962 by Mariner 2 probe. Measured speed - $300-700 km/s$. 

1990 ESA/NASA Ulysses mission detected strong correlation of solar wind to number of sunspots. It also detected a variation in power (with solar activity) of a factor of $4$.

\subsubsection{Fluid model of the solar wind}

Note that this neglects kinetic effects and is not completely correct!

We'll work, obviously, in spherical coordinates. Assume solar wind velocity $v(r)$ is purely radial and dependent only on radial distance. Our parameters then are pressure $P$, temperature $T$, density $\rho$.

Use mass conservation and equation of motion, including force of solar gravity. Assume negligible $\crossproduct{j}{B}$ term. Assume also that since we're looking for stationary solutions, time derivatives are negligible.

Assume also a hydrostatic solution - zero wind velocity. Then we can use the ideal gas law for protons and electrons together (neglecting electron mass compared to ion mass for averaging). Assume isothermal gas (constant temperature). It turns out that the pressure falls inversely with radius. The result is rather nonphysical because we get a finite pressure at inifinity... so the hydrostatic solution is a failure.

Roll back the zero wind velocity, but carry on with the isothermal plasma assumption. 

Define sound speed as $v_s^2=\frac{K_b T}{\mu}$, $\mu$ being the average particle mass.

Define also a critical radius $r_g = \frac{G M_{sum}}{2 v_s^2}$

\subsubsection{Solar wind}

\[ \oneover{v} \d{v}{r} (\frac{v^2}{v_s^2}-1) = \frac{2}{r}(1-\frac{r_c}{r}) \]

At small $r$, $1-r_c/r<0$ and for assumed small starting velocity $(v/v_s)^2-1 < 0$. Thus $\d{v}{r}>0$ - the plasma accelerates and the solar wind starts to blow.

The phase space is divided into four regions by the two critical values $v_s$, $r_c$.

\begin{enumerate}
\item If a trajectory reaches $v=v_s$ at $r<r_c$ it accelerates while turning back. This is nonphysical.

\item If $v<v_s$ at $r=r_c$, the wind accelerates until the critical radius (minimum derivative and then begins slowing down. This is called a subsonic breeze.

\item If $v_s$ at $r=r_c$, the wind keeps accelerating afterwards. The acceleration is always positive, though may become small (asympotically zero). This is a supersonic wind.
\end{enumerate}

\subsubsection{Typical values for solar wind}

\[ r_c = \frac{M_sum G}{2 v_s^2} \]

\[ v_s = \sqrt{\frac{k_b T}{\mu}} \]

Average mass $\mu = 0.6 m_p$ due to helium presence

Temperature $1.5e6 K$.

Boltzmann constant $k_b = 1.38e-23 J/K$.

This gives us $v_s = 1.4e5 m/s$, $r_c = 3.4e9 m = 4.5 R_{sun}$. This is where the wind becomes supersonic.

We can use these velocities with the equation of motion (assuming isothermality), integrating over trajectories to get energy conservation (a version of Bernoulli's theorem). Our calculated values give us initial conditions.

\subsubsection{Solar mass loss due to solar wind}

It turns out to be $1.58e9 kg/s$ - negligible compared to solar mass.

Plasma density in solar corona - $10e14 m_p$. 

\subsubsection{Solar wind at Earth position}

$1 AU \sim 214 R_0 \sim 48 r_c$, which gives a velocity of $4 v_s = 5.6e5 m/s$. About 500 kilometers a second. It needs a good couple days to reach Earth and a good couple \textbf{months} to reach the outer planets.

The solar wind strikes the magnetic field and, through the frozen flux theorem, pushes it into the outer Solar System.

The interplanetary magnetic field is weak in magnitude, but it influences our magnetosphere. As the solar wind reaches our magnetosphere, it has a density of about $3-8$ ions per cubic centimeter. The magnetosphere carves a cavity free from those ions and deflects them. This forms a $17 km$-thich bow shock layer at a distance of about $90000 km$ away from the Earth.

Some solar wind plasma penetrates the bow shock and forms the \textbf{magnetosheath}.The \textbf{magnetopause} is the abrupt boundary where the planetary magnetic field pressure and solar wind pressure equalize.

The solar wind creates a neutral sheet behind the Earth, where B lines in opposite directions lie close to each other. This is just the condition for magnetic reconnection! This happens often during CMEs as the tail gets essentially cut off by the increased solar wind pressure. This pumps energetic particles into our atmosphere, producing auroras.

\end{document}
