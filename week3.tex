\documentclass[PlasmaNotes.tex]{subfiles}
\begin{document}
\setcounter{section}{2}
\section{Week 3. Fluid description of plasmas, with Paolo Ricci}
\subsection{From Vlasov to two-fluid}

\begin{itemize}
\item We'll turn the computationally expensive kinetic model into a simple fluid model (can be solved with partial differential function)
\item We'll derive fluid quantities (pressure, etc) from distribution functions
\item We'll take a moment for moments of the kinetic equation
\item We'll apply the continuity equation (conservation of mass etc)
\item ... and we'll get the two fluid model!
\end{itemize}

\subsubsection{Fluid quantities from the distribution function}
In the fluid model we only really look at quantities that depend on position. This means we'll have to integrate out all velocity dependence.

Number density - the integral of distribution function over all velocities. Simple. We just care about the density \emph{at a location}.
\[ \]

For any function, it's average value is the integral of the function weighted by distribution function, normalized by dividing the integral by the number density of the species.
\[ \]

Examples:
\begin{itemize}
\item Average fluid velocity
\[ \]
\item Average kinetic energy density
\[ \]
\item Pressure \textbf{tensor} (as pressure may depend on direction)
\[ \]
\end{itemize}

\subsubsection{Examples of fluid averaged quantities}
\begin{enumerate}
\item Beam density. Distribution function with uniform spread in $1D$ positions and a delta function dependence on velocity (only one velocity in the whole beam).

The density ends up being just $n_0$ because of the Dirac delta velocity dependence.

The fluid velocity is just the velocity of the particles, as the delta function times velocity gives us just the particle velocity upon integrating out.

The kinetic energy density is just $m v_0^2/2$, obviously.

The pressure tensor ends up as zero! Both the vector quantities end up being zero as $\v{v} = \v{v_0}$. Reasonable. If everything's moving at the same velocity there's not going to be anything pushing anything else.

\item Maxwellian velocity distribution with uniform position distribution
\[ \]

Number density is just $n_0$, as the velocity distribution is normalized to one anyway.

Average fluid velocity - gaussian average value $v_0$. 

Average kinetic energy density: kinetic energy from average velocity plus $\frac{3}{2} k_b T$ (thermal velocity)

Pressure tensor is a diagonal matrix.
\[ \]
\end{enumerate}

\subsubsection{The moments of the kinetic equation}

We take the Boltzmann equation:

\[ \]

Shift the collision term to the left side

\[BOLT =  \]
And take averages - `moments' - of that by integrating $BOLT$ it with a weight $g$. This will give us the following equations:
\begin{itemize}
\item Continuity - $g=1$
\item Momentum - $g=mv$
\item Energy - $g=mv^2/2$
\end{itemize}
The continuity equation will be done here.

\subsubsection{The continuity equation}

\[ \]

\begin{itemize}
\item The first term ends up as just the time derivative of average number density
\[ \]
\item The first term is the time derivative of average fluid velocity times average number density
\[ \]

\item The third term: the forces can be brought under the velocity derivative. We then use the divergence theorem in velocity space and since particles aren't going to have infinite velocities, the surface integral over the region in velocity space is zero.
\[ \]

\item The fourth collision term is zero on the grounds that 'collisions do not create nor destroy particles'. Not sure how that works...
\end{itemize}

On the whole, we get the neat continuity equation:
%Get this a square!
\[ \]

\subsubsection{The two fluid model!}

We treat electrons as one fluid, and electrons as another, separate fluid. The two fluids are in the same region in space and interacting with each other (not much of a plasma otherwise!).

We take the continuity equation derived above

\[ \]

We also take the momentum equation, which was left for the student to derive

\[ \]

Where $\d{}{t}$ is the total time derivative (streaming included)

\[ \]

And $R$ is the collision term

\[ \]

We also take the energy equation

Where $T_s$ is the temperature factor related to spread of velocity

\[ \]

And $q_s$ is the heat flux vector, the kinetic energy transported by the velocity of the fluid

\[ \]

And $Q_s$ is the heat generated by viscous forces (basically friction, collisions?)

\[ \]

The system of equation is \textbf{not closed}, we need an expression for the heat flux, which requires knowing the distribution function. This is a \emph{closure problem}. Closures can be difficult to find and people are actively sciencing that as hard as they can. The problem is that you'd need to use the distribution function to get quantities we'd like to use instead of the distribution function.

Coupling with Maxwell equations is achieved by noting that the charge density is just the species number density times species charge, summed over all species. Current density, likewise, is the sum of species number density times species charge times average species velocity. With these two (and a set of initial conditions), Maxwell equations can be solved for the forces on our fluids.

\subsubsection{Two fluid simulation of plasma turbulence}

Does it make sense to use the two-fluid model for simulations to describe plasmas instead of, y'know, particles?\footnote{\href{https://www.youtube.com/watch?v=5sGKuoBnTn0&feature=youtu.be}{If the author of the notes seems prejudiced in favor of using particles, it may be related to the fact that particle simulations are \textbf{awesome}.}}

It does make sense - it's relatively easy to find a good closure to close the set of equations. It's helpful when plasmas are \emph{fairly collisional}. In that regime, the distribution function is usually close to Maxwellian and that can help find a closure.

Two fluid simulations are especially helpful in the periphery of magnetic fusion devices, where plasmas are relatively cold. This actually helps with turbulence, and turbulence is HARD to deal with:

\begin{quotation}
	`I am an old man now, and when I die and go to heaven there are two matters on which I hope for enlightenment. One is quantum electrodynamics, and the other is the turbulent motion of fluids. And about the former I am rather optimistic.'
	
- Horace Lamb -
\end{quotation}

\end{document}