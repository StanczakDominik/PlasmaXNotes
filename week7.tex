\documentclass[PlasmaNotes.tex]{subfiles}
\begin{document}
\setcounter{section}{6}
\let\oldexp\exp
\renewcommand{\exp}[1]{\oldexp(#1)}
\newcommand{\png}[1]{\begin{center}\includegraphics{#1}\end{center}}
\newcommand{\goesto}{\rightarrow}
\section{Week 7. Thermonuclear fusion: an overview, with Ambrogio Fasoli}

\subsection{The basics}

\subsubsection{Why fusion?}

HDI correlated with electricity use per capita (though Canada is using almost three times as much as the Netherlands so there's room for improvement there). So to develop humankind we need better power sources.

We have to fight global warming, stop using fossil fuels and save this polar bear.

\png{thisbear.PNG}

We also have to make sure the power sources are safe for the population. Energy sources that endanger people are \textbf{bad}.

And fusion fulfils all of those criteria!

\subsubsection{How does it work}

\png{bindingenergycurve.PNG}

Two possible approaches to nuclear energy generation - split really heavy nuclei, like uranium, or merge small nuclei. Note that as the slope is sharp on the left, so you get a lot of energy per each fusion.

\subsubsection{Fusion reactions for a terrestrial reactor}

Three possibilities: DT, DHe3, DD fuel cycles.

At low kinetic energies coulomb repulsion deflects particles. This prevents fusion from occuring (essentially interception by atomic forces). High kinetic energies are equivalent to saying we need high temperature.

You don't actually need $380 keV$ to get into the potential well on the left - you can do with $10 keV$ due to tunneling (a quantum effect).

\png{7FuelCycles.PNG}

Cross section times average velocity over distribution functions (AKA \textbf{reactivity}) - this is a parameter that describes how easy it is for particles to fuse. The DT reaction has the largest one at that for the smallest temperatures, so we're focusing on that for now.

At these temperatures ($10 keV$ or so), pretty much all matter is in the plasma state! So we're gonna need to go through this to understand that.

\textbf{Thermo}nuclear fusion - as Coulomb collisions have a humongous cross section, any monoenergetic beam would be scattered in a plasma before the particles have time to fuse. Thus, an important condition is that our plasma is in a \textbf{thermalized} Maxwellian distribution (this would be mean that the plasma is kind of in a stationary state).

\subsubsection{The DT cycle}

\png{7DTCycle.PNG}

This is the current approach to terrestrial fusion. Deuterium is ubiquitous in the oceans. Tritium has a half-life of $12.5$ years and has to be \textbf{bred} from lithium. That can also be found in the ocean.

We look into the $Li-6$ reaction due to its large cross-section. Smack a neutron into the $Li-6$ atom and one $T$ and one $He-4$ pops out. Lithium can be supplied by using a \textbf{blanket} mounted on the inside wall of the vacuum vessel.

\subsubsection{Schematic of a fusion reactor}

We inject deuterium into the reactor and lithium into the blanket. The tritium stays in a closed system. Helium can be bottled and made into balloons or sold at high prices. The energy is removed from the system by a conventional fluid based system.

\subsubsection{Why fusion is neat}

Fusion energy density: $3.5e8 MJ/kg$. Compare to $30 MJ/kg$ for coal, $50 MJ/kg$ for oil, $8.5e7 MJ/kg$ for uranium fission. $9e10 MJ/kg$ for $E=mc^2$ total conversion.

The fuels are virtually inexhaustible. Lithium specifically can be mined (one example - Nevada), both lithium and deuterium can be obtained from sea water. This is so uniformly spread over the Earth that you could almost call it open source.

The environmental impact? No greenhouse gases whatsoever, helium is neat. The radioactive elements, depending on the materials used, take a short time to become safe.

It is impossible to weaponize a fusion reactor. You can't lose control of the reaction (no criticality) and there's very little fuel in the reactor at any given time (about $1g$) .

\png{7NeatReactorPicture.PNG}

\begin{center}
\textbf{Let's go do that.}
\end{center}

\subsection{Power balance in reactors}

Each DT reaction gives $3.5 MeV$ to the $\alpha$ and $14.1 MeC$ to the neutron, for a total of $17.6 MeV$.

Fusion power density for a particular relative velocity (or energy). $n_D, n_T$ are densities of deuterium and tritium. $\sigma$ is the reaction cross-section.
\begin{equation}
R_{DT}(v) \Delta E_f = (n_D n_T \sigma_{DT} (v) v) \Delta E_f
\end{equation}

We can integrate that over velocity distributions for both D and T.

Assuming neutrality ($n_D=n_T=n_e/2=n/2$), neglecting impurities and density of alpha particles:
\begin{equation}
\text{Power density} = n_D n_T \avg{\sigma v}_{DT} \Delta E_f = \oneover{4} n^2 \avg{\sigma v}_{DT} \Delta E_f
\end{equation}

To produce $1GW$ of fusion power at $T=20keV$, $n=5e20 m^{-3}$, use tabularized data for the average $\avg{\sigma v}$. Divide desired fusion power by fusion power density. The result: $14 m^3$ - apparently a reasonable volume.

\subsection{Generalities}

For steady state operation - conservation of energy. The losses include radiation, convection and conduction (`direct losses'). Our power input is whatever we do to heat the plasma and the fusion power produced inside.

The radiative lossses include bremsstrahlung - electrons are accelerated in an electric field, producing x-ray radiation (which can pass through the plasma and the containment vessel - thus escaping). Power density of that $\sim n^2 T_e^{1/2}$.

The direct losses are described by the energy confinement time $\tau_E$. Power density of that $=3nT/\tau_E$.

\subsection{Calculation of power balance - breakeven}

Def. physics fusion gain $Q=\text{Fusion power/Input power} = P_f/P_in$. A reactor needs $Q>1$. At $Q=1$ we have a breakeven situation. $P_in + P_\alpha = P_{losses} = P_{direct losses} + P_{bremsstrahlung}$, so we get $Q=\frac{P_f}{P_{direct losses} + P_{bremsstrahlung} - P_{\alpha}}$.

Inserting the previously derived expressions:

\begin{equation}
 n \tau_E \geq \frac{12 T}{\frac{6}{5} \avg{\sigma v}_{DT} \Delta E_F + 4 A Z_{eff} T^{1/2}}
\end{equation}

At large temperatures (over $1 keV$), we can neglect the second term to get

\begin{equation}
\oneover{n \tau_E} = f(T) = \frac{\avg{\sigma v}_{DT} \Delta E_F}{10 T}
\end{equation}

\png{7breakevencurve.png}

Reactors can work in the green area above the breakeven curve.

\subsubsection{Ignition}

This is the limit where all the heating the plasma needs comes from the fusion reactors. This would be the Sun on Earth scenario. $P_F/P_{in} = +\infty$ as $P_{in} \goesto 0$

$P_in = 0$ means that $P_\alpha \geq P_{losses} \sim P_{dl}$ (we neglect bremsstrahlung).

$n \tau_E \text{at ignition} \geq \frac{12 T}{\avg{\sigma v}_{DT} \Delta E_\alpha} = 6/f(t)$. It's just a higher parabola!

\png{7ignition.png}

The burning plasma regime is when alpha particle heating is greater than the input heating, $Q \geq 5$.

\subsubsection{Engineering fusion gain}

Def. $Q_E$ - ratio of electric power output to input, $\eta_e$ - efficiency of plasma heating electrical power supply, $\eta_t$ - efficiency of thermal fusion power conversion into energy. $\eta = \eta_e \eta_t$, $Q_E=\frac{\eta(P_f + P_{in}) - P_{in}}{P_{in}} = \eta Q - (1-\eta)$. $Q \gg Q_E$, sadly.

\subsection{Plasma confinement approaches}

We need both $n \tau_E = 10^{20} m^{-3} s$ and $T\geq 10 keV$. Magnetic confinement has $\tau_E$ about $1s$. Inertial confinement goes with $10^{-9}s$ and much higher densities.



\subsection{Inertial confinement}

We consider the ion confinement time $\tau_i = R/c_s$, $R$ being the pellet radius and $c_s$ the speed of sound, at which the ions tend to move.

We also define the heating time $\tau_h=\frac{\text{energy in pellet}}{\text{power deposited on surface}}=\frac{3nT 4\pi R^3/3}{F 4 \pi R^2} = \frac{n T R}{F}$, F being the energy flux

We must have heating faster than the confinement time. $\tau_h < \tau_i \rightarrow F >  m T^{3/2} / \sqrt{m_i}$. The size of the pellet cancels out. Note that we neglected all compressional effects (shock waves...).

For a standard example, $F$ is about $5e19 W/m^2$. This means lasers. Really big lasers.

\subsubsection{ICF energy balance}

Laser inputs energy $E_L$ - I took the liberty of changing notation. Thermal energy is scaled by efficiency $\epsilon_c$: $E_{th}=E_L \epsilon_c$. This produces fusion energy $E_f$ and that process also has a finite efficiency in conversion to electricity, $\epsilon_{th}$. We can also recirculate some of the fusion energy to drive the reactor.

\png{7icfloop.png}

It turns out there's a simple condition for the laser energy sufficient for fusion. For typical values (solid deuterium), and relatively low efficiencies ($5\%$, $10\%$) it turns out to give $10^{15} J$ - $50$ times the Hiroshima bomb. We can currently achieve $2 MJ$. This requires a density of $3000 n_0$, $n_0$ being solid deuterium density. This means we have to have a pressure of $3000 n_0 T$, which is $2e17$ Pascal. That's a lot. Can't do that just by firing lasers.

\subsubsection{The rocket effect}

Generate an implosion by launching a lot of energy into the external surface. Conservation of momentum will cause an inward motion of the external layers and this will compress the plasma.

Pressure $P_{rocket} = \frac{\d{(mv)}{t}}{\text{surface}} \sim \frac{V \d{m}{t}}{S}$ at constant $v$

The energy flux $F$, at constant $v$, turns out to be $\frac{v^2 dm}{2 S dt}$ and we can derive the $\d{m}{t} = \frac{2 S F}{v^2}$.

Cancelling the terms, we get $\frac{P_{rocket}}{P_{laser}} = \frac{2c}{v}$. This is much bigger than 1 and thus a considerable boost

\subsubsection{Additional ICF effects}

The shockwave, if it goes symmetrically into the center, helps increase pressure.

Alpha production can also help heat the core, but the pellet must be larger than alpha mean free path.

However, instabilities, as always, occur and the laser light can't be too intense. Also, too much density means the laser beam actually gets reflected (plasma frequency cutoff occurs)

\subsubsection{Summary of physics}

\begin{itemize}
 \item Radiation phase
 
 Surface heated by laser, gets plasmified
 \item Blowoff phase
 
 This is where the surface is blown off and the rest is pushed inside.
 
 \item Implosion phase
 
 Thermal energy is transported to the center, the core is heated
 
 \item Thernomuclear burning
 
\end{itemize}

The challenges here are as follows: The core cannot be heated before the shock wave hits it (separate it from the rest somehow?). Imposes constraints on laser pulse timing and pellet design.

Also, there's the question of hydrodynamic stability:

\png{7rayleightaylorpellet.png}

The pellet must be very symmetric and the beams have to be really carefully aligned. Shown above is a mixing of the DT fuel and core material due to the Rayleigh Taylor instability - we don't really want that.

\subsubsection{Direct and indirect drive}

The direct drive is the simpler one - we just smack the pellet with lasers, as symmetrically as possible. There's multiple layers - a low density ablator on the outside, a ball of solid or liquid fuel and a gas layer.

The indirect drive is intended to lower the symmetry requirements. We have a small chamber - the so called hohlraum - around the pellet. It receives laser beams through two holes and acts as a black body source of x-rays. This is inherently more symmetric than injecting the pellet with laser energy.

The Lawrence Livermore Laboratory's National Ignition Facility uses 192 lasers to achieve pulse energy of $2 MJ$, with about $500 TW$ of power. These fire UV light ($352 nm$).

\subsubsection{Engineering challenges}

Cost, efficiency, reliability. Difficult to find materials for first vacuum chamber wall. Capsules are really complex (3d printing, perhaps?).

A reactor would need to fire a few times a second to function as a power generator.

\subsection{Magnetic confinement}

Confine particles by making them rotate around magnetic field lines. Macroscopic confinement times allow lower densities.

There's many schemes for this:
\begin{itemize}
 \item Linear devices - such as magnetic mirrors
 \item Toroidal configurations
 \item Stellarators
\end{itemize}

\subsubsection{Simple toroidal device example}

A simple toroidal field has curvature and an intensity gradient. This causes drifts - positive charges go to the top, negative charges to the bottom. This causes first an E field, then an ambient magnetic field and this means loss of confinement.

What we do is impose a poloidal field. Through their motions particles stay in regions of different drifts directions.

There's the so called hoop force which tends to move field lines to the outer region of the tokamak. To avoid this, we add a vertical field which counters that.

To sum up:
\begin{itemize}
 \item Toroidal field coils
 \item Plasma current generates poloidal field
 \item Outer poloidal field coils that generate the vertical field, can also help positioning 
\end{itemize}

How to drive the current in the plasma? That's a neat idea - it's basically a transformer. The plasma is the secondary circuit and the primary circuit is a coil inside the core - center - of the tokamak. Drive AC through that and you drive AC through the plasma by Faraday's law.

\subsubsection{Progress}
Density times confinement time with respect to temperature - we're steadily moving towards the high energy multiplication and sustained burn region. $Q = 5$ is getting really close now. ITER should have $Q$ of about $10$ - a reactor needs about $30$ or $40$. TFTR and JET actually got close to breakeven by significantly increasing the confinement time. ITER aims to push that to several hundreds of second.

ITER is being built in Cadarache in the south of France. A really global project.\footnote{Really makes you wonder why the victory condition in Civilization games is the space race to Alpha Centauri, huh?}

And now, back to our regularly scheduled programme on magnetic confinement.
5

\subsection{Simple design of a magnetic fusion reactor}

Constraints on the design result from nuclear physics and engineering.

We put lithium in the blanket, which also collects the heat (usual water energy transfer system).

Tritium is produced in the blanket - has to be kept in a closed system, as it's used in the DT plasma for fusion purposes.

Around the blanket we need magnets for B field generation.

We want $1<Q<\infty$, between breakeven and ignition. $n\tau_E$ is thus between $1e20$ and $6e20 m^{-3}s$. $T\geq 10keV$.

We want to minimize the cost of the reactor and the requirements on plasma performance. This means minimizing $\tau_E$ and $\beta$ - ratio of plasma pressure and magnetic pressure.

We'll be optimizing:
\begin{itemize}
 \item the size of the reactor
 \item reactor geometry
 \item magnitude of $B$ field
 \item $n$
 \item $T$
 \item $\tau_E$
\end{itemize}

Our constraints are those resulting from both physics and engineering.

\subsubsection{Simplified geometry for magnetic fusion reactor}

Assume toroidal general shape, circular cross section. Plasma on the inside, radius $a$. Blanket of thickness $b$. Magnet thickness $c$.

From engineering:

We want to produce $P_E=1GW$ for now.

Wall loading $L_W$ - this tells us how much power we can be depositing on the wall. Usually about $5 MW/m^2$.

The magnets have to be superconducting in a certain region in BJT space - in practice this says that we can only have $13 T$ at the coil.


From nuclear physics:

Value of fusion rate determined by fusion cross-section.

Processes in the blanket also have their cross-sections:

Neutron multiplication (has to occur before breeding tritium, has to occur at the same rate).

Neutron slowing down - at high neutron energies tritium production cross-section decreases.

Note that we only consider lithium 6 - at low temperatures the cross section for Li-7 is negligibly small. We don't really have to worry about that, it seems (at least at this basic level).

We must also shield the coils from the neutrons - don't want those to become radioactive, of course!

\subsubsection{Neutron physics - blanket and shielt thickness}

For each process, the thickness required is the inverse of number density of targets times cross section.

Neutron multiplication can be done on beryllium. Thickness comes out to be $13 cm$.

Neutrum slowing down can be done on lithium itself - that's $20 cm$.

Tritium breding at room temperature $0.025 eV$ with $7.5\% Li^6$ turns out to be $0.2 cm$. Note that the $7.5\%$ purity of lithium comes into the number density. So we don't have to add more than necessary for slowing down, it seems.

For coil shielding, say we want to reduce the flux $100$ times. Assume only the slowdown layer (which is the largest) - we need $1m$.

Total thickness about $1.2m$ - it's not direct addition as layers overlap in function.

\subsubsection{Minimisation of cost of electricity}

Minimising reactor cost over power is equivalent in this kind of work to minimizing volume of complex systems over power. This means - everything but the plasma, that's not something you build.

Volume of torus $2 \pi R_0 \times \pi ((a+b+c)^2-a^2)$. Electrical power produces is $1/4$ efficiency of power input times sum of energies in alpha, neutron and tritium (in blanket!) production, times density squared, times DT cross-section, times plasma volume ($2\pi R_0 \pi a^2$)

We use the wall loading constraint to get $R_0$ - we take the $5 MW/m^2$ value times plasma surface area.

Neutron power is fusion power over efficiency in transformation into electric energy, times fraction of neutron energy over total produced energy.

From the equality between the latter two pops out (upon assuming $40\% = \eta_t$) an expression for $R_0$.

We can now take the complex system volume over power fraction - the electrical power actually drops out - and we're left with an expression:

\[
 \frac{V}{P} = 0.8 \frac{(a+b+c)^2-a^2}{a L_W^{max}}
\]

Thus the better materials we have (more wall loading), the lower the volume and cost. $b$ is already set - due to blanket physics - so we're optimizing in 2-space, over $a$ and $c$.

\subsubsection{Magnet thickness}

That should also be minimum. There's the $\crossproduct{J}{B}$ force on the magnet (superconducting coil) and it has to be able to withstand the stress. A calculation with tensile stress gives $c=\frac{2\alpha}{1-\alpha}(a+b)$, alpha being $\frac{B_{coil}^2}{4\mu_0 \sigma_{max}}$, $\sigma_{max}$ being the maximum stress the magnet can endure.

It's difficult to reach high $\beta$ in the plasma. So we need the maximum field at the coil - $13 T$ (at higher values they stop superconducting, at least with today's technology).

Maximum stress is about $200 MPa$. $\alpha \sim 0.1$. So $c=0.22(a+b)$ - it simplifies out of our optimization problem! It's a one variable problem, in this simplified version.

\[
 \frac{V}{P} = 0.8 \frac{(a+b+c)^2-a^2}{a L_W^{max}}
\]

Taking the derivative we get $a=\sqrt{3} b$. $c=0.22(\sqrt{3}+1)b=0.6b$. $b=1.2m$, $a=2.1m$, $c=0.7m$.

\subsubsection{Resulting reactor geometry and plasma parameters}

Taking $L_W^{max} = 4.5 MW/m^2$ we get $R_0 = 4.2m$. The plasma surface becomes $350m^2$, volume $365m^2$. The power density in that - power of alphas and neutrons over volume - $6 MW/m^3$. $B_0$ being the magnetic field in the plasma, at the major radius. $B \propto 1/r$ and at the coil it's $13 T$, so we'll get $2.7 T$ at the major radius, assuming coil located at $R_0-a-b$.

Assuming $\tau_E = 2s$, $n=1e20 m^-3$, $T=15 keV$, $\beta$ turns out to be $= \frac{n T}{B_0^2/2\mu_0} = 8\%$.

\end{document}