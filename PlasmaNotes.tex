\input{header.tex}
\title{Notes from PlasmaX}
\author{Dominik `Perfi' Stańczak}

\begin{document}
\maketitle

These notes will not be $100\%$ comprehensive, as I'm making them mainly for my own use. However, if you spot any mistakes, feel free to catch me on the forums and I'll fix any mistakes. 

\subsection{The notes' repository}

The repository for these notes is maintained \href{https://github.com/StanczakDominik/PlasmaXNotes}{here}. Feel free to add any changes to the .tex files (if you don't feel like compiling the pdf don't worry, I've got that covered).

\subsection{The course wiki and errata pages}

The course's got a wiki page which you can check out \href{https://courses.edx.org/courses/course-v1:EPFLx+PlasmaX+2T2015/wiki/EPFLx.PlasmaX.2T2015/}{here}. I try to incorporate corrections from the \href{https://courses.edx.org/courses/course-v1:EPFLx+PlasmaX+2T2015/wiki/EPFLx.PlasmaX.2T2015/errata/}{errata page} into these notes, but obviously I may miss some of those. If you find anything wrong, add that in there and/or say so on the forums.

\subsection{Qni's Julia package}
Julia along with the Equations package can be used as an open source alternative to Matlab.

Official Julia website: http://julialang.org/

Equations repository: https://github.com/jhlq/Equations.jl

To install the package type, inside Julia: Pkg.clone("https://github.com/jhlq/Equations.jl")

Example use in the form of solutions to course assignments is available in notebook format in the PlasmaXNotes repository, in the examples directory of Equations and as a html at http://artai.co/Plasma.html

The Equations package is currently in a developing stage so please submit any encountered issues.

\newpage
\subfile{week1}

\newpage

\subfile{week2}

\newpage

\subfile{week3}

\newpage

\subfile{week4}

\newpage

\subfile{week5}

\newpage

\subfile{week6}

\newpage

\subfile{week7}

\newpage

\end{document}