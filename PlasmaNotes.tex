% ***********************************************************
% ******************* PHYSICS HEADER ************************
% ***********************************************************
% Version 2
\documentclass[11pt]{article} 
\usepackage{amsmath} % AMS Math Package
\usepackage[utf8]{inputenc}
\usepackage[T1]{fontenc}
\usepackage{subfiles}
\usepackage{hyperref}
\usepackage{amsthm} % Theorem Formatting
\usepackage{amssymb}	% Math symbols such as \mathbb
\usepackage{graphicx} % Allows for eps images
\usepackage{multicol} % Allows for multiple columns
\usepackage[dvips,letterpaper,margin=0.75in,bottom=0.5in]{geometry}
 % Sets margins and page size
\pagestyle{empty} % Removes page numbers
\makeatletter % Need for anything that contains an @ command 
\renewcommand{\maketitle} % Redefine maketitle to conserve space
{ \begingroup \vskip 10pt \begin{center} \large {\bf \@title}
	\vskip 10pt \large \@author \hskip 20pt \@date \end{center}
  \vskip 10pt \endgroup \setcounter{footnote}{0} }
\makeatother % End of region containing @ commands
\renewcommand{\labelenumi}{(\alph{enumi})} % Use letters for enumerate
% \DeclareMathOperator{\Sample}{Sample}
\let\vaccent=\v % rename builtin command \v{} to \vaccent{}
\renewcommand{\v}[1]{\ensuremath{\mathbf{#1}}} % for vectors
\newcommand{\gv}[1]{\ensuremath{\mbox{\boldmath$ #1 $}}} 
% for vectors of Greek letters
\newcommand{\uv}[1]{\ensuremath{\mathbf{\hat{#1}}}} % for unit vector
\newcommand{\abs}[1]{\left| #1 \right|} % for absolute value
\newcommand{\avg}[1]{\left< #1 \right>} % for average
\let\underdot=\d % rename builtin command \d{} to \underdot{}
\renewcommand{\d}[2]{\frac{d #1}{d #2}} % for derivatives
\newcommand{\dd}[2]{\frac{d^2 #1}{d #2^2}} % for double derivatives
\newcommand{\pd}[2]{\frac{\partial #1}{\partial #2}} 
% for partial derivatives
\newcommand{\pdd}[2]{\frac{\partial^2 #1}{\partial #2^2}} 
% for double partial derivatives
\newcommand{\pdc}[3]{\left( \frac{\partial #1}{\partial #2}
 \right)_{#3}} % for thermodynamic partial derivatives
\newcommand{\ket}[1]{\left| #1 \right>} % for Dirac bras
\newcommand{\bra}[1]{\left< #1 \right|} % for Dirac kets
\newcommand{\braket}[2]{\left< #1 \vphantom{#2} \right|
 \left. #2 \vphantom{#1} \right>} % for Dirac brackets
\newcommand{\matrixel}[3]{\left< #1 \vphantom{#2#3} \right|
 #2 \left| #3 \vphantom{#1#2} \right>} % for Dirac matrix elements
\newcommand{\grad}[1]{\gv{\nabla} #1} % for gradient
\let\divsymb=\div % rename builtin command \div to \divsymb
\renewcommand{\div}[1]{\gv{\nabla} \cdot \v{#1}} % for divergence
\newcommand{\curl}[1]{\gv{\nabla} \times \v{#1}} % for curl

\newcommand{\laplace}[1]{\Delta #1} % for laplacian
\newcommand{\eps}{\epsilon_0} % electric permittivity
\newcommand{\oneover}[1]{\frac{1}{#1}} % 1/thing
\newcommand{\dotproduct}[2]{\v{#1} \cdot \v{#2}} % for dotproduct
\newcommand{\crossproduct}[2]{\v{#1} \times \v{#2}} % for crossproduct
\newcommand{\fourierv}[1]{\tilde{\v{#1}}}
%aliases
\newcommand{\para}{\parallel}
\newcommand{\rot}[1]{\curl{#1}}

\let\baraccent=\= % rename builtin command \= to \baraccent
\renewcommand{\=}[1]{\stackrel{#1}{=}} % for putting numbers above =
\newtheorem{prop}{Proposition}
\newtheorem{thm}{Theorem}[section]
\newtheorem{lem}[thm]{Lemma}
\theoremstyle{definition}
\newtheorem{dfn}{Definition}
\theoremstyle{remark}
\newtheorem*{rmk}{Remark}

% ***********************************************************
% ********************** END HEADER *************************
% ***********************************************************
\title{Notes from PlasmaX}
\author{Dominik `Perfi' Stańczak}

\begin{document}
\maketitle

These notes will not be $100\%$ comprehensive, as I'm making them mainly for my own use. However, if you spot any mistakes, feel free to catch me on the forums and I'll fix any mistakes. 

\subsection{The notes' repository}

The repository for these notes is maintained \href{https://github.com/StanczakDominik/PlasmaXNotes}{here}. Feel free to add any changes to the .tex files (if you don't feel like compiling the pdf don't worry, I've got that covered).

\subsection{The course wiki and errata pages}

The course's got a wiki page which you can check out \href{https://courses.edx.org/courses/course-v1:EPFLx+PlasmaX+2T2015/wiki/EPFLx.PlasmaX.2T2015/}{here}. I try to incorporate corrections from the \href{https://courses.edx.org/courses/course-v1:EPFLx+PlasmaX+2T2015/wiki/EPFLx.PlasmaX.2T2015/errata/}{errata page} into these notes, but obviously I may miss some of those. If you find anything wrong, add that in there and/or say so on the forums.

\subsection{Qni's Julia package}
Julia along with the Equations package can be used as an open source alternative to Matlab.

Official Julia website: http://julialang.org/

Equations repository: https://github.com/jhlq/Equations.jl

To install the package type, inside Julia: Pkg.clone("https://github.com/jhlq/Equations.jl")

Example use in the form of solutions to course assignments is available in notebook format in the PlasmaXNotes repository, in the examples directory of Equations and as a html at http://artai.co/Plasma.html

The Equations package is currently in a developing stage so please submit any encountered issues.

\newpage
\subfile{week1}

\newpage

\subfile{week2}

\newpage

\subfile{week3}

\newpage

\subfile{week4}

\newpage
\end{document}